%===============================================================================
\section{Shimming Procedure} \label{sec:Shimming-Procedure}
%===============================================================================


%===============================================================================
\subsection{Calibration}
%===============================================================================

%===============================================================================
\subsection{OPERA}
%===============================================================================

%===============================================================================
\subsection{Knobs}
%===============================================================================

%===============================================================================
\subsection{Laminations}
%===============================================================================

\begin{itemize}
   \item Motivation and physics 
         \begin{itemize}
            \item Foils modeled as nearly fully-saturated dipoles ($\mu \sim$ 0.6--0.7) 
            \item Modeling accounts for magnetic images  
            \item Fill in valleys of field map 
            \item Need small angular extent ($\sim 1^{\circ}$ effects); fine control over multipoles 
         \end{itemize}  
   \item Picket Fences 
         \begin{itemize}
            \item Talk about reasoning: fixed probe FIDs unacceptable when 
                  using regular orientation of foils 
         \end{itemize} 
   \item Calibration 
         \begin{itemize} 
            \item Discuss calibration batches (production, laser cut, etc)  
         \end{itemize}
   \item Mass Production 
         \begin{itemize} 
            \item Cutting by hand to start (testing phase); move to laser cutting 
            \item Laser cutting: wide array of widths, can produce picket fences easily 
         \end{itemize}  
   \item Assembly
         \begin{itemize}
            \item Preprocessing: all foils cleaned in alcohol, manual histogramming into bins
            \item Lamination layout
                  \begin{itemize}
                     \item Lamination panel: single piece is the full length/width of one pole piece in azimuth 
                     \item Radially segmented into sections of 3: optimized for multipoles 
                     \item Foil distribution determined by code 
                  \end{itemize}   
            \item Procedure for assembly
                  \begin{itemize}
                     \item Templates for left, center, right poles 
		     \item Foil shopping
                     \item Double-sided tape (3M 9485PC) placed on G10 base
                  \end{itemize}
         \end{itemize} 
   \item Installation 
         \begin{itemize}
            \item Epoxy to pole surface 
            \item Alignment to $\sim 10$'' downstream of yoke-yoke boundary  
            \item Brace with jack stands, wood cutouts to fit pole pieces; assembly in sections of 8 poles   
         \end{itemize} 
   \item Data analysis and results   
         \begin{itemize}  
            \item Plots showing few poles worth of laminations 
            \item Compare with/without full lamination installation
            \item Statistics: peak-to-peak variation, integration over full ring 
            \item Compare to BNL   
         \end{itemize} 
\end{itemize} 

%===============================================================================
\subsection{Surface Coils}
%===============================================================================

%===============================================================================
\subsection{Optimization}
%===============================================================================


