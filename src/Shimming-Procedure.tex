%===============================================================================
\section{Shimming Procedure} \label{sec:Shimming-Procedure}
%===============================================================================


%===============================================================================
\subsection{Calibration}
%===============================================================================

%===============================================================================
\subsection{OPERA}
%===============================================================================

%===============================================================================
\subsection{Knobs}
%===============================================================================

\begin{itemize}
  \item Overview of Shimming Knobs
    \begin{itemize}
      \item Pole Adjustments - normalize steel planes      
      \item Top Hats - wide angle, dipole
      \item Wedge Shims - medium angle, dipole/quadrupole
      \item Edge(let) Shims - wide angle, multipoles
      \item Laminations - new design, small angle, low multipoles
    \end{itemize}
  \item Calibration values in summary table
  \item Illustrate basic model for a shim
\end{itemize}

%===============================================================================           
\subsubsection{Pole Moves}
%=============================================================================== 

\begin{itemize}
  \item Hardware description
    \begin{itemize}
      \item Steel composition and uniformity specification
      \item Stabilizing ``feet''
      \item Create two annuli forming a highly uniform gap
    \end{itemize}
  \item Effects on B-field
    \begin{itemize}
      \item Gap determines first order field strength
      \item Gradients in gap cause multipoles
    \end{itemize}
  \item Adjustment procedure
    \begin{itemize}
      \item Remove pole plugs
      \item Techs use crane to lower onto blocks
      \item Adjust stack of shims under pole feet to 12um
    \end{itemize}
  \item Implentation
    \begin{itemize}
      \item Gap improvements
      \item Tilt improvements
      \item Field improvements
    \end{itemize}
\end{itemize}

%===============================================================================
\subsubsection{Top Hats}
%===============================================================================

\begin{itemize}
  \item Hardware description - plates of iron on top and bottom of ring
  \item Effects on B-field
    \begin{itemize}
      \item Change inductance of C-magnet locally in azimuth
      \item No effects on higher order multipoles, only dipole
    \end{itemize}
  \item Adjustment procedure
    \begin{itemize}
      \item Techs loose plates
      \item Adjust stack of shims under top hats to 10s of mils
    \end{itemize}
  \item Field improvements
\end{itemize}

%===============================================================================
\subsubsection{Wedges}
%=============================================================================== 

\begin{itemize}
  \item Hardware description
    \begin{itemize}
      \item Iron wedges that live between yokes and poles
      \item 12 wedges per pole
      \item Freedom to move radially
    \end{itemize}
  \item Effects on B-field
    \begin{itemize}
      \item Locally adjusts amount of steel at a given radius
      \item Affects dipole and quarupole fields
    \end{itemize}
  \item Adjustment procedure
    \begin{itemize}
      \item Measure initial radial position of all wedges
      \item Adjust radial position by turning screw at inner radius
      \item Record final radial positions
    \end{itemize}
  \item Field improvements
\end{itemize}

%===============================================================================
\subsubsection{Edge Shims and Edgelet Laminations}
%=============================================================================== 

\begin{itemize}
  \item Hardware description
    \begin{itemize}
      \item rails of iron at edges of uniform pole surface
      \item located at lower, upper, lower and outer positions
      \item edgelet cut size from 10 degrees to 1 degree
    \end{itemize}
  \item Effects on B-field
    \begin{itemize}
      \item Adding/removing steel influences field
      \item Multiple positions gives handle on multipoles
      \item Edgelets give even more localized handle on multipoles
    \end{itemize}
  \item Adjustment procedure
    \begin{itemize}
      \item Order over-sized
      \item Predict required change and grind down
      \item Repeat as necessary
    \end{itemize}
  \item Implementations
    \begin{itemize}
      \item Edge shim test - deemed unnecessary in the end
      \item Edgelet test - less powerful knob than laminations
    \end{itemize}
\end{itemize}


%===============================================================================
\subsection{Laminations}
%===============================================================================

\begin{itemize}
   \item Motivation and physics 
         \begin{itemize}
            \item Foils modeled as nearly fully-saturated dipoles ($\mu \sim$ 0.6--0.7) 
            \item Modeling accounts for magnetic images  
            \item Fill in valleys of field map 
            \item Need small angular extent ($\sim 1^{\circ}$ effects); fine control over multipoles 
         \end{itemize}  
   \item Picket Fences 
         \begin{itemize}
            \item Talk about reasoning: fixed probe FIDs unacceptable when 
                  using regular orientation of foils 
         \end{itemize} 
   \item Calibration 
         \begin{itemize} 
            \item Discuss calibration batches (production, laser cut, etc)  
         \end{itemize}
   \item Mass Production 
         \begin{itemize} 
            \item Cutting by hand to start (testing phase); move to laser cutting 
            \item Laser cutting: wide array of widths, can produce picket fences easily 
         \end{itemize}  
   \item Assembly
         \begin{itemize}
            \item Preprocessing: all foils cleaned in alcohol, manual histogramming into bins
            \item Lamination layout
                  \begin{itemize}
                     \item Lamination panel: single piece is the full length/width of one pole piece in azimuth 
                     \item Radially segmented into sections of 3: optimized for multipoles 
                     \item Foil distribution determined by code 
                  \end{itemize}   
            \item Procedure for assembly
                  \begin{itemize}
                     \item Templates for left, center, right poles 
		     \item Foil shopping
                     \item Double-sided tape (3M 9485PC) placed on G10 base
                  \end{itemize}
         \end{itemize} 
   \item Installation 
         \begin{itemize}
            \item Epoxy to pole surface 
            \item Alignment to $\sim 10$'' downstream of yoke-yoke boundary  
            \item Brace with jack stands, wood cutouts to fit pole pieces; assembly in sections of 8 poles   
         \end{itemize} 
   \item Data analysis and results   
         \begin{itemize}  
            \item Plots showing few poles worth of laminations 
            \item Compare with/without full lamination installation
            \item Statistics: peak-to-peak variation, integration over full ring 
            \item Compare to BNL   
         \end{itemize} 
\end{itemize} 

%===============================================================================
\subsection{Surface Coils}
%===============================================================================
\begin{itemize}
  \item Hardware and Motivation
      \begin{itemize}
          \item 100 concentric coils above and below vacuum chambers
          \item Current range $\pm$2.5A 
          \item Current configurations used to create opposing higher order average multipole moments
          \item Current configuration used to cancel average radial field
          \item Goal: All multipoles moments less than 1 ppm, variation across storage region less than 1 ppm
      \end{itemize}
  \item Radial Dependence of Currents and Calibration
      \begin{itemize}
          \item In free space, curl of $\vec{B}$ is 0, so magnetic field can be represented as a gradient
          \item Start with general expression for gradient and calculate x and y components of B field
          \item Assume that poles have large permeability, get boundary condition on normal $\vec{H}$ field
          \item Use boundary condition to derive radial dependence of current density that creates each multipole
          \item Table of current configurations for each multipoles
          \item Tested each configuration with max current 2A. Calculated effect of each configuration on all multipoles
          \item All verified linearity when changing max current of each configuration
      \end{itemize}
  \item Optimization and Results
      \begin{itemize}
          \item Optimize the currents by minimizing $|Ax-y|$
          \item A is stack of multipole effects of each current configuration
          \item x is the maximum current associated with each configuration
          \item y is the multipoles to minimize
          \item Show plots with results
      \end{itemize}
\end{itemize}


%===============================================================================
\subsection{Optimization}
%===============================================================================

\begin{itemize}
  \item Mathematial model
    \begin{itemize}
      \item Build up overall problem by add basic models
      \item Add in weights to account for different multipoles 
    \end{itemize}
  \item Minimization routine
    \begin{itemize}
      \item Iterative least squares routine
      \item Build in switches to turn off certain knobs 
      \item Model validation
      \item Model caveats
    \end{itemize}
\end{itemize}
