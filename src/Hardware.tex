%=============================================================================== 
\section{Hardware}
\label{sec:Hardware}
%=============================================================================== 

%=============================================================================== 
\subsection{Magnet Design and Shimming Kit}
%=============================================================================== 

%=============================================================================== 
\subsection{Instrumentation}
%=============================================================================== 

%=============================================================================== 
\subsubsection{Commercial Tools}
%=============================================================================== 

%=============================================================================== 
\subsubsection{Shimming Cart}
%=============================================================================== 

%=============================================================================== 
\subsubsection{Tilt Sensors}
%=============================================================================== 

\begin{itemize}
    \item Hardware
        \begin{itemize}
            \item Narrow range electrolyic tilt sensors from Frederick's company (Part \#0719-3701-99)
            \item 16-bit signal conditioning board (Part \#1-6200-012)
            \item Mounted on a three point contact system to measure radial and azimuthal tilt. Include picture
        \end{itemize}
    \item Calibration
        \begin{itemize}
            \item Inserted shims of varying thickness under 1 point of contact
            \item Recorded tilt (in bits) from both tilt sensors for each configuration
            \item Fit recorded tilt vs. expected tilt (in radians)
            \item result: 1 bit is 1.3 urad
        \end{itemize}
    \item Uses
        \begin{itemize}
            \item Large input to pole foot adjustments
            \item Measured tilts across pole face. Used to adjust relative inner-outer radial tilt, minimize quadrupole moment
            \item Measured tilts on both sides of a pole-pole boundary, as well as across boundary, to calculate pole step
            \item Mounted on Hall probe platform to monitor tilt and ensure repeatable positioning after a rotations
        \end{itemize}
\end{itemize}

%=============================================================================== 
\subsubsection{Hall Probes}
%=============================================================================== 

\begin{itemize}
    \item Hardware and Purpose
        \begin{itemize}
            \item NMR probes measure field magnitude, not field direction. Can't be used to measure radial/longitudinal field components
            \item Hall probes are sensitive to both field magnitude and direction
            \item Used ASensor HE-244T Hall probes, which have a reported ultra low planar Hall coefficient
        \end{itemize}
    \item Hall Probe Measurements
        \begin{itemize}
            \item Hall probe measurements get a contribution from normal Hall effect, planar Hall effect, offset voltage (include equation)
            \item By rotating Hall probe about normal axis, were able to measure planar Hall coefficient. Found that it was negligibly small and could be safely ignored 
            \item Offset voltage minimized by taking difference of 2 measurements with Hall probe rotated by 180 degrees about vertical axis
            \item Built platform that allowed for 180/90 degree rotation for radial and longitudinal field measurements
            \item Platform tilt could be adjusted at urad level (monitored by tilt sensors)
            \item Platform height and radial position can be adjusted to study fields all over muon storage region
        \end{itemize}
\end{itemize}
